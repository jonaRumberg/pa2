\section{Einleitung}

\subsection{Kontext und Relevanz des Themas}
Das Thema der künstlichen Intelligenz ist aktuell so heiß diskutiert wie kein zweites. 
Seitdem die Firma OpenAI Anfang des Jahres 2023 das Sprachmodell "Chat-GPT" veröffentlichte, steht das Thema prominent in der Öffentlichkeit. 
Viele Firmen haben seither künstliche Intelligenz als strategisch wichtiges Mittel erkannt und treiben die Entwicklung von Geschäftsanwendungen,
die von KI profitieren stark voran. 

Ein solcher Anwendungsfall ist die natürlichsprachliche Datenbanksuche. Das Konzept hierbei ist es, komplexe
Filter und Sortierungen durch eine Texteingabe zu ersetzen, welche daraufhin von einer künstlichen Intelligenz verarbeitet wird. Diese macht es
möglich, aus einer simplen Texteingabe semantische Bedeutung herauszulesen und entsprechende Datenbankeinträge zurückzugeben. Der Mehrwert für den
Endnutzer ergibt sich daraus, dass er sich keine Gedanken über die technische Umsetzungs seiner Suche mehr machen muss, sondern das gewünschte
Ergebnis im Optimalfall durch natürlichsprachliche Formulierung seiner Anfrage erhält und so auch sehr einfach komplexe Anfragen formulieren kann.

\subsection{Ziel der Arbeit}
Das Ziel der Arbeit ist es, einen bestehenden Prototypen für das beschriebene Szenario zu optimieren. Dafür sollen zuerst theoretische Grundlagen zum
Thema künstliche Intelligenz geklärt werden und dabei speziell auf den Begriff des Embeddings eingegangen werden. Darauf folgend sollen Zielgrößen
bestimmt werden, mit denen die Güte des Systems verlässlich beurteilen zu können und im Anschluss verschiedene Optimierungsansätze vorgestellt und
bewertet werden. Der Fokus dieser Optimierung werden die vorher besprochenen Embeddings sein. Denkbar sind der Vergleich verschiedener KI-Modelle
zur Erstellung von Embeddings oder verschiedene Arten, wie Embeddings generiert werden. Als Zielgrößen kommen dabei sowohl qualitative Kennzahlen, 
als auch Performance oder Kosten in Frage.
Schließlich soll mit der Arbeit die Frage beantwortet werden, was im Kontext der natürlichsprachlichen Datenbanksuche ein geeigneter Weg ist, 
Embeddings zu erstellen um ein optimales Ergebnis zu erreichen.
