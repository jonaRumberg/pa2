\section{Einleitung}

\subsection{Kontext und Relevanz des Themas}
Keine Entwicklung der Welt der IT ist aktuell so viel besprochen wie die der künstlichen Intelligenz. Speziell durch den Aufstieg von generativer KI hat sich das Thema zu einer geradezu gesamtgesellschaftlich relevanten Entwicklung herangebildet.
Maßgeblich angestoßen durch die Veröffentlichung von OpenAI's GPT-3 Modell, welches in der Lage ist, Texte zu generieren, die von menschlichen Texten nur noch schwer zu unterscheiden sind, hat sich die öffentliche Aufmerksamkeit auf die Möglichkeiten von generativer KI gerichtet. Die Konzepte und Technologien, die hinter diesen Entwicklungen stehen, sind dabei nicht unbedingt neu, eine breitere Verfügbarkeit von Rechenleistung und Trainingsdaten haben jedoch den entscheidenen Anstoß für die neue Leistungsfähigkeit dieser Modelle gegeben.
KI ist also das das Thema der Stunde und als strategisch relevantes Thema für Unternehmen und Organisationen nicht mehr wegzudenken.

\subsection{Ziel der Arbeit}
Die vorliegende Arbeit beschäftigt sich mit einem konkreten Anwendungsfall von KI in der Praxis. Untersucht wird ein Beispiel, in dem eine semantische Datenbanksuche auf einem Materialstammdatensatz durchgeführt wird. Der Mehrwert dieses Systems liegt dabei in der Möglichkeit für Anwender, die Datenbank auf natürlichsprachliche Weise zu durchsuchen, ohne dabei auf die spezifischen Suchbegriffe und -syntaxen achten zu müssen, die in traditionellen Datenbanksystemen notwendig sind und gleichzeit von einem gewissen semantischen Verständnis des Suchsystems profitieren zu können.
Die Technik aus dem Feld der KI, die für dieses System zum Tragen kommt sind sogenannte \textit{word embeddings}, die es ermöglichen, Worte in einem Vektorraum abzubilden und so semantische Ähnlichkeiten zwischen Wörtern zu berechnen. Dieses Konzept wird in der Arbeit genauer erläutert die Effektivität verschiedener Techniken zur Erstellung von embeddings im konkreten Anwendungsfall beleuchtet.
Das Ziel der Arbeit ist es, die Technik im Anwedungsfall zu erläutern, verschiedene Methoden und Modelle zu beleuchten und eine datengestützte Entscheidungsgrundlage für die Bewertung von word embeddings in semantischen Suchsystemen zu schaffen.


